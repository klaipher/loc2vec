\documentclass[12pt,a4paper]{article}
\usepackage[utf8]{inputenc}
\usepackage[english]{babel}
\usepackage{amsmath}
\usepackage{amsfonts}
\usepackage{amssymb}
\usepackage{graphicx}
\usepackage{float}
\usepackage{hyperref}
\usepackage{geometry}
\usepackage{booktabs}
\usepackage{caption}
\usepackage{subcaption}
\usepackage{fancyhdr}
\usepackage{titlesec}
\usepackage[numbers,sort&compress]{natbib}

% Page geometry
\geometry{margin=2.5cm}

% Header and footer
\pagestyle{fancy}
\fancyhf{}
\fancyhead[L]{Loc2Vec for Kyiv City}
\fancyhead[R]{UCU Machine Learning Project}
\fancyfoot[C]{\thepage}
\setlength{\headheight}{14.5pt}

% Title formatting
\titleformat{\section}{\large\bfseries}{\thesection}{1em}{}
\titleformat{\subsection}{\normalsize\bfseries}{\thesubsection}{1em}{}

% Hyperref setup
\hypersetup{
    colorlinks=true,
    linkcolor=blue,
    filecolor=magenta,
    urlcolor=cyan,
    citecolor=blue
}

\title{
    \textbf{Location Embeddings with Triplet-Loss Networks: \\
    A Loc2Vec Implementation for Kyiv City}
}

\author{
    \textit{Machine Learning Project Report} \\
    \textit{Ukrainian Catholic University} \\
    \vspace{0.5cm}
    \textit{Team Members:} \\
    Dmytro Miskevych, Stanislav Pavliuk, Bohdan Dhzus \\
    \vspace{0.3cm}
    \textit{Course:} Machine Learning \\
    \textit{Academic Year:} 2024-2025
}

\date{\today}

\begin{document}

\maketitle

\begin{abstract}
This report presents our implementation of the Loc2Vec methodology for learning location embeddings using triplet-loss networks, specifically applied to Kyiv city. Based on the original paper by Sentiance, we developed a system that transforms geographic locations into meaningful vector representations by leveraging OpenStreetMap data and convolutional neural networks. Our approach incorporates transfer learning techniques using pre-trained models including EfficientNet, ResNet, and MobileNetV3 to improve embedding quality and training efficiency. The project demonstrates the effectiveness of spatial representation learning for understanding geographic semantics and relationships within urban environments, with potential applications in venue mapping, location recommendation, and urban planning.
\end{abstract}

\tableofcontents
\newpage

\section{Introduction}

In today's digital world, understanding where things happen has become incredibly important. From GPS navigation to urban planning and even targeted advertising, location-based services are everywhere. But here's the thing - most current approaches to representing locations are pretty basic. They either just use coordinates (latitude and longitude) or rely on manually crafted features, which don't really capture what makes different places similar or different.

Think about it this way: two places might be right next to each other on a map but be completely different in character - like a quiet residential street next to a busy industrial area. On the flip side, two shopping districts on opposite sides of a city might feel very similar even though they're geographically far apart. Traditional location representation methods struggle with this kind of semantic understanding.

The Loc2Vec methodology, originally developed by Sentiance \cite{sentiance2018loc2vec}, addresses this challenge by learning dense vector representations of geographical locations that capture both spatial and semantic relationships. The approach transforms locations into embeddings using convolutional neural networks trained with triplet loss, where the network learns to place semantically similar locations closer together in the embedding space.

\subsection{Problem Statement}

For our project, we decided to implement and adapt the Loc2Vec methodology specifically for Kyiv, Ukraine's capital city. This wasn't just a straightforward implementation though - we faced several interesting challenges. First, we wanted our model to truly understand what makes different areas of Kyiv unique, whether it's the historic city center, Soviet-era housing blocks, or modern business districts. We also needed to figure out how to encode the complex spatial relationships between different parts of the city.

What made our project particularly interesting was our decision to experiment with modern deep learning architectures that weren't available when the original Loc2Vec paper was published. We tested whether transfer learning with architectures like EfficientNet, ResNet, and MobileNetV3 could improve upon the original results. Finally, we had to adapt the entire approach to work well with Kyiv's specific urban characteristics and the available OpenStreetMap data for the city.

\subsection{Project Objectives}

Our main goals for this project were pretty ambitious. We wanted to build a complete working implementation of Loc2Vec from scratch, using OpenStreetMap data specifically for Kyiv. But we didn't stop there - we were curious about whether modern neural network architectures could do better than the original approach, so we planned to test EfficientNet, ResNet, and MobileNetV3 as potential encoders.

We were also really interested in transfer learning. The idea was to see if starting with pre-trained models (instead of training from scratch) could make our training faster and produce better embeddings. To make sure we could properly evaluate our work, we needed to set up comprehensive metrics to measure how good our location embeddings actually were. And of course, we wanted to show that these embeddings could be useful for real applications, not just academic exercises.

\subsection{Contributions}

Our work makes several contributions to the field of spatial representation learning:

\begin{itemize}
    \item A complete open-source implementation of Loc2Vec adapted for Kyiv city using modern deep learning frameworks
    \item Comprehensive comparison of different CNN architectures for location embedding learning
    \item Investigation of transfer learning techniques in the context of geographical data
    \item Analysis of the learned embeddings' ability to capture urban semantics and spatial relationships
    \item Practical insights into the application of location embeddings for real-world problems
\end{itemize}

\subsection{Report Structure}

This report is organized as follows: Section \ref{sec:related_work} reviews relevant literature in spatial representation learning and embedding techniques. Section \ref{sec:data} describes our data collection and preprocessing pipeline for Kyiv city. Section \ref{sec:approach} details our implementation methodology and architectural choices. Section \ref{sec:results} presents experimental results and analysis. Finally, Section \ref{sec:further_steps} discusses future directions and potential improvements.

\subsection{Location Embeddings and Spatial Representation Learning}

The concept of learning distributed representations for geographical locations has emerged as a powerful approach for various location-based tasks. Early work in this area focused on matrix factorization techniques and collaborative filtering approaches adapted for spatial data \cite{zheng2010collaborative}.

The introduction of Word2Vec \cite{mikolov2013word2vec} and its success in natural language processing inspired researchers to develop similar approaches for geographical data. Several variants have been proposed:

\begin{itemize}
    \item \textbf{Place2Vec} \cite{yan2017place2vec}: Learns place embeddings by treating location sequences as sentences in NLP
    \item \textbf{Location2Vec} \cite{liu2017location2vec}: Uses check-in sequences to learn venue embeddings
    \item \textbf{Geography2Vec} \cite{zhang2018geography2vec}: Incorporates geographical constraints into embedding learning
\end{itemize}

\subsection{Venue Mapping and Point-of-Interest Prediction}

Venue mapping represents a specific application of location intelligence where the goal is to determine the most likely venue a user is visiting given noisy location measurements. Traditional approaches include:

\textbf{Distance-based methods}: Simple nearest neighbor approaches that select the closest venue to a GPS coordinate. While computationally efficient, these methods fail in dense urban areas with multiple nearby venues.

\textbf{Probabilistic approaches}: Methods that model location uncertainty and venue popularity using probabilistic frameworks \cite{krumm2008predicting}. These approaches often incorporate temporal patterns and user behavior models.

\textbf{Machine learning approaches}: Feature-based methods that use various location characteristics (distance, venue type, temporal patterns) as input to traditional ML algorithms \cite{lian2014geomf}.

\subsection{Triplet Loss and Metric Learning}

The triplet loss function, originally proposed for face recognition \cite{schroff2015facenet}, has become a fundamental technique for learning embedding spaces where similarity relationships are preserved. The key idea is to train networks using triplets of examples: an anchor, a positive (similar) example, and a negative (dissimilar) example.

The triplet loss is defined as:
\begin{equation}
L = \max(0, \|f(x_a) - f(x_p)\|^2 - \|f(x_a) - f(x_n)\|^2 + \alpha)
\end{equation}

where $f(\cdot)$ is the embedding function, $x_a$, $x_p$, $x_n$ are anchor, positive, and negative examples respectively, and $\alpha$ is the margin parameter.

Applications of triplet loss in spatial domains include:
\begin{itemize}
    \item Image retrieval for location recognition \cite{arandjelovic2016netvlad}
    \item Satellite image analysis \cite{hu2018deep}
    \item Urban planning and land use classification \cite{jean2016combining}
\end{itemize}

\subsection{Transfer Learning in Computer Vision}

Transfer learning has revolutionized computer vision by enabling the adaptation of models pre-trained on large datasets (like ImageNet) to specific domains with limited data. Recent architectural advances have led to several families of efficient CNN architectures:

\textbf{EfficientNet} \cite{tan2019efficientnet}: Systematically scales network depth, width, and resolution using compound scaling, achieving state-of-the-art efficiency.

\textbf{ResNet} \cite{he2016resnet}: Introduces residual connections to enable training of very deep networks, addressing the vanishing gradient problem.

\textbf{MobileNetv3} \cite{howard2019mobilenetv3}: Optimized for mobile deployment with depthwise separable convolutions and neural architecture search.

These architectures have been successfully applied to various spatial analysis tasks, though their application to location embedding and venue mapping remains underexplored.

\subsection{OpenStreetMap and Geographical Data Processing}

OpenStreetMap (OSM) \cite{haklay2008openstreetmap} provides a rich, crowdsourced geographical database that has enabled numerous research applications. Key advantages include:
\begin{itemize}
    \item Comprehensive global coverage with detailed urban features
    \item Rich semantic labeling of geographical objects
    \item Regular updates and community maintenance
    \item Free access and open license
\end{itemize}

Tools for processing OSM data include PostGIS for spatial queries, GDAL for data conversion, and various Python libraries (OSMnx, Fiona, Shapely) for programmatic access.

\subsection{Gap Analysis}

While existing work has made significant progress in location embedding and venue mapping, several gaps remain:

\begin{enumerate}
    \item \textbf{Limited architectural exploration}: Most location embedding approaches use relatively simple CNN architectures without exploring modern efficient architectures
    \item \textbf{Regional bias}: Many studies focus on well-studied regions (US, Western Europe) with limited exploration of Eastern European urban characteristics
    \item \textbf{Transfer learning evaluation}: Systematic evaluation of transfer learning approaches for geographical data is lacking
    \item \textbf{Computational efficiency}: Limited analysis of trade-offs between model accuracy and computational requirements for mobile deployment
\end{enumerate}

Our work addresses these gaps by providing a comprehensive evaluation of modern CNN architectures for location embedding in the context of Kyiv city, with particular attention to transfer learning and computational efficiency considerations.

\section{Data}
\label{sec:data}

This section describes our data collection and preprocessing pipeline for creating location embeddings specific to Kyiv city. We detail the data sources, preprocessing steps, and the challenges encountered in adapting the methodology to the Ukrainian capital.

\subsection{Data Sources}

\subsubsection{OpenStreetMap Data}

We decided to use OpenStreetMap (OSM) as our main data source, which turned out to be a great choice. OSM is basically Wikipedia for maps - it's a collaborative project where people around the world contribute geographical information. For Kyiv specifically, we were able to extract a ton of useful data including the complete road network (everything from tiny residential streets to major highways), building outlines for different types of structures, and land use information that tells us where the parks, industrial areas, and residential zones are located.

We also got detailed information about points of interest like restaurants, shops, schools, hospitals, and cultural sites, plus all the transportation infrastructure including metro stations, bus stops, and railway stations. One of the coolest things about working with Kyiv data was getting to map natural features like the Dnieper River, which runs right through the city, along with various lakes and green spaces.

What really helped us was that Kyiv has a pretty active OpenStreetMap community, so the data quality was surprisingly good. This gave us detailed coverage of the city's infrastructure and amenities that we might not have gotten in other cities.

\subsubsection{Administrative Boundaries}

We incorporated administrative boundary data to understand the city's structure:

\begin{itemize}
    \item \textbf{District Boundaries:} Kyiv's 10 administrative districts (raions)
    \item \textbf{Neighborhood Boundaries:} Historical and informal neighborhood divisions
    \item \textbf{City Limits:} Official boundaries of Kyiv and surrounding areas
\end{itemize}

\subsection{Geographic Coverage}

Our dataset covers the greater Kyiv metropolitan area, including:

\begin{itemize}
    \item \textbf{Central Kyiv:} Historical city center and main commercial districts
    \item \textbf{Residential Areas:} Soviet-era housing districts and modern developments
    \item \textbf{Industrial Zones:} Manufacturing and industrial areas
    \item \textbf{Suburban Areas:} Surrounding suburbs and satellite towns
    \item \textbf{Natural Areas:} Parks, forests, and the Dnieper River corridor
\end{itemize}

The total area covered spans approximately 1,500 square kilometers, encompassing the diverse urban landscape of Kyiv from dense city centers to suburban and semi-rural areas.

\subsection{Data Preprocessing Pipeline}

\subsubsection{Geographic Information System Setup}

We established a comprehensive GIS processing pipeline:

\begin{enumerate}
    \item \textbf{Database Setup:} Installed PostgreSQL with PostGIS extensions for spatial data operations
    \item \textbf{OSM Import:} Used osm2pgsql to import Kyiv OSM data into the spatial database
    \item \textbf{Data Validation:} Performed quality checks and cleaned inconsistent or incomplete data
    \item \textbf{Indexing:} Created spatial indices for efficient geographic queries
\end{enumerate}

\subsubsection{Map Tile Generation}

Following the original Loc2Vec approach, we developed a map rasterization system:

\textbf{Rasterization Framework:} We used Mapnik with custom stylesheets to convert vector geographic data into raster tiles. This process generates multi-channel tensors where each channel represents a specific type of geographical feature.

\textbf{Channel Configuration:} Our 12-channel tensor representation includes:
\begin{itemize}
    \item Channel 1: Major roads and highways
    \item Channel 2: Secondary and residential roads
    \item Channel 3: Pedestrian paths and cycling routes
    \item Channel 4: Residential buildings
    \item Channel 5: Commercial and office buildings
    \item Channel 6: Industrial buildings
    \item Channel 7: Water bodies (rivers, lakes)
    \item Channel 8: Green spaces (parks, forests)
    \item Channel 9: Public transportation infrastructure
    \item Channel 10: Points of interest and amenities
    \item Channel 11: Administrative boundaries
    \item Channel 12: Land use classifications
\end{itemize}

\textbf{Tile Specifications:} Each tile represents a 500m × 500m area at 256×256 pixel resolution, providing sufficient detail to capture local geographical structures while maintaining computational efficiency.

\subsection{Data Augmentation}

To increase dataset diversity and improve model generalization, we implemented several data augmentation techniques:

\subsubsection{Geometric Transformations}
\begin{itemize}
    \item \textbf{Rotation:} Random rotations from 0° to 360° to ensure orientation invariance
    \item \textbf{Translation:} Random horizontal and vertical shifts up to 50 pixels
    \item \textbf{Scaling:} Minor zoom variations (±10\%) to account for different detail levels
\end{itemize}

\subsubsection{Channel-wise Augmentation}
\begin{itemize}
    \item \textbf{Noise Addition:} Gaussian noise to simulate GPS uncertainty and mapping errors
    \item \textbf{Channel Dropout:} Randomly disabling certain channels to improve robustness
    \item \textbf{Intensity Variations:} Adjusting channel intensities to simulate different data qualities
\end{itemize}

\subsection{Dataset Statistics}

Our final dataset contains:

\begin{itemize}
    \item \textbf{Total Locations:} 50,000 unique coordinate pairs covering Kyiv
    \item \textbf{Tile Coverage:} Complete coverage of urban areas with 80\% coverage of suburban areas
    \item \textbf{Training Set:} 35,000 locations (70\%)
    \item \textbf{Validation Set:} 7,500 locations (15\%)
    \item \textbf{Test Set:} 7,500 locations (15\%)
    \item \textbf{Augmented Samples:} 5x augmentation resulting in 250,000 training samples
\end{itemize}

\subsection{Triplet Generation Strategy}

For triplet loss training, we developed a sophisticated triplet mining strategy:

\subsubsection{Positive Pair Generation}
Positive pairs are locations that should have similar embeddings:
\begin{itemize}
    \item \textbf{Spatial Proximity:} Locations within 100-200m radius
    \item \textbf{Semantic Similarity:} Locations with similar land use patterns
    \item \textbf{Infrastructure Similarity:} Areas with comparable road networks and building densities
\end{itemize}

\subsubsection{Negative Pair Generation}
Negative pairs are semantically different locations:
\begin{itemize}
    \item \textbf{Different Districts:} Locations from different administrative areas
    \item \textbf{Contrasting Land Use:} Industrial vs. residential, urban vs. natural areas
    \item \textbf{Infrastructure Differences:} Dense urban areas vs. sparse suburban areas
\end{itemize}

\subsubsection{Hard Negative Mining}
We implemented online hard negative mining to focus training on challenging examples:
\begin{itemize}
    \item \textbf{Semi-hard Negatives:} Negatives that are closer to the anchor than the positive
    \item \textbf{Dynamic Selection:} Updating negative selection based on current model performance
    \item \textbf{Balanced Sampling:} Ensuring representation across different geographical areas
\end{itemize}

\subsection{Data Quality and Challenges}

\subsubsection{Quality Assurance}
We implemented several quality control measures:
\begin{itemize}
    \item \textbf{Visual Inspection:} Manual review of generated tiles for accuracy
    \item \textbf{Consistency Checks:} Verification of channel alignment and data integrity
    \item \textbf{Boundary Validation:} Ensuring proper handling of city boundaries and water bodies
\end{itemize}

\subsubsection{Challenges Encountered}
Several challenges were addressed during data preparation:
\begin{itemize}
    \item \textbf{Data Completeness:} Some areas had incomplete OSM coverage requiring additional validation
    \item \textbf{Urban Complexity:} Kyiv's diverse architecture and Soviet-era planning required careful feature representation
    \item \textbf{Seasonal Variations:} OSM data represents a snapshot in time, not accounting for seasonal changes
    \item \textbf{Scale Consistency:} Ensuring consistent representation across different urban densities
\end{itemize}

The resulting dataset provides a comprehensive foundation for learning meaningful location embeddings that capture the unique characteristics of Kyiv's urban landscape.

\section{Approach}
\label{sec:approach}

This section details our implementation methodology, architectural design decisions, and experimental setup for adapting the Loc2Vec approach to Kyiv city with modern deep learning techniques.

\subsection{System Architecture Overview}

We designed our implementation with a modular architecture that would be flexible and easy to extend. The core of our system consists of several key components that work together. Our data pipeline handles all the OSM data processing, generates map tiles, and mines triplets for training. The model architecture is built around CNN encoders, where we can swap in different backbone architectures like EfficientNet, ResNet, or MobileNetV3.

For training, we built a framework around triplet loss optimization with some advanced techniques we'll describe later. We also created a comprehensive evaluation system to measure how good our embeddings actually are, and developed visualization tools that let us interactively explore and analyze the learned embeddings - this turned out to be really helpful for understanding what our model was learning.

\subsection{Neural Network Architecture}

\subsubsection{Base Encoder Design}

Our encoder architecture transforms 12-channel map tiles into dense vector representations. The base architecture follows these principles:

\textbf{Input Processing:} Each 12-channel tile (256×256×12) undergoes initial processing to normalize channel intensities and handle missing data.

\textbf{Feature Extraction:} Convolutional layers extract hierarchical features from local patterns to global spatial structures.

\textbf{Embedding Generation:} A final dense layer produces fixed-size embeddings (typically 128 or 256 dimensions) that serve as location representations.

\subsubsection{Backbone Architectures}

We decided to experiment with three different modern CNN architectures as our encoder backbone, each with its own strengths.

\textbf{EfficientNet-B0} was our first choice because it's known for providing an excellent balance between accuracy and efficiency through something called compound scaling. To make it work with our 12-channel map tiles (instead of the usual 3-channel RGB images), we had to modify the first convolutional layer, remove the classification head that was designed for ImageNet, and add our own global average pooling followed by a dense layer to generate embeddings.

\textbf{ResNet-18 and ResNet-34} were natural candidates because they're proven architectures with residual connections that make training deep networks much easier. We made similar modifications - adapting the input layer for our 12 channels, replacing the final fully connected layer with our embedding layer, and tweaking some batch normalization settings to work better with spatial data.

\textbf{MobileNetV3-Small} was interesting because it's designed to be really efficient, which could be useful if we wanted to deploy our model on mobile devices or in resource-constrained environments. We made the same basic modifications but also focused on optimizations that would keep inference fast while maintaining good accuracy.

\subsection{Transfer Learning Strategy}

\subsubsection{Pre-training Adaptation}

Since our input differs significantly from ImageNet (12 channels vs. 3 RGB channels), we developed several strategies for leveraging pre-trained weights:

\textbf{Channel Initialization:} We initialized the first convolutional layer weights by:
\begin{itemize}
    \item Replicating RGB weights across relevant channels (e.g., road networks, buildings)
    \item Random initialization for channels without clear RGB analogies (e.g., administrative boundaries)
    \item Weighted combinations for channels representing mixed features
\end{itemize}

\textbf{Progressive Unfreezing:} We employed a progressive training strategy:
\begin{enumerate}
    \item Stage 1: Freeze backbone, train only the embedding layer
    \item Stage 2: Unfreeze final layers, fine-tune with reduced learning rate
    \item Stage 3: Full network fine-tuning with careful learning rate scheduling
\end{enumerate}

\textbf{Domain Adaptation:} To bridge the gap between natural images and map data:
\begin{itemize}
    \item Applied specialized data augmentation techniques
    \item Used domain-specific normalization strategies
    \item Implemented gradual adaptation through curriculum learning
\end{itemize}

\subsection{Training Methodology}

\subsubsection{Triplet Loss Implementation}

We implemented an enhanced version of triplet loss with several improvements:

\textbf{Loss Function:}
\begin{equation}
L_{triplet} = \max(0, ||f(a) - f(p)||_2^2 - ||f(a) - f(n)||_2^2 + \alpha)
\end{equation}

where $\alpha$ is the margin parameter, set to 0.2 after hyperparameter tuning.

\textbf{Batch Construction:} Each training batch contains carefully constructed triplets:
\begin{itemize}
    \item Batch size of 32 triplets (96 samples total)
    \item Online hard negative mining within each batch
    \item Balanced sampling across different geographical regions
\end{itemize}

\textbf{Mining Strategy:} We implemented sophisticated triplet mining:
\begin{itemize}
    \item \textbf{Hard Negative Mining:} Select negatives that are closest to the anchor
    \item \textbf{Semi-hard Mining:} Choose negatives that are closer than the positive plus margin
    \item \textbf{Curriculum Learning:} Gradually increase mining difficulty during training
\end{itemize}

\subsubsection{Training Configuration}

\textbf{Optimization:}
\begin{itemize}
    \item Optimizer: Adam with initial learning rate 1e-4
    \item Learning rate schedule: Cosine annealing with warm restarts
    \item Weight decay: 1e-5 for regularization
    \item Gradient clipping: Maximum norm of 1.0
\end{itemize}

\textbf{Training Protocol:}
\begin{itemize}
    \item Total epochs: 100 with early stopping based on validation loss
    \item Validation frequency: Every 5 epochs
    \item Model checkpointing: Save best model based on validation metrics
    \item Mixed precision training: For memory efficiency and speed
\end{itemize}

\subsection{Data Loading and Augmentation}

\subsubsection{Efficient Data Pipeline}

We designed an efficient data loading pipeline to handle the large dataset:

\textbf{Multi-threaded Loading:} Parallel data loading with 8 worker processes to minimize I/O bottlenecks.

\textbf{Memory Management:} Implemented intelligent caching strategies:
\begin{itemize}
    \item LRU cache for frequently accessed tiles
    \item On-demand tile generation for augmented samples
    \item Memory mapping for large tile datasets
\end{itemize}

\textbf{Preprocessing Pipeline:} Real-time preprocessing including:
\begin{itemize}
    \item Channel normalization based on dataset statistics
    \item Invalid data handling (missing channels, corrupted tiles)
    \item Dynamic triplet generation based on current model state
\end{itemize}

\subsubsection{Advanced Augmentation}

Beyond basic geometric transformations, we implemented domain-specific augmentations:

\textbf{Geographical Augmentations:}
\begin{itemize}
    \item Seasonal variations simulation (changing vegetation coverage)
    \item Construction simulation (adding/removing buildings)
    \item Traffic density variations (road usage intensity changes)
\end{itemize}

\textbf{Noise and Corruption:}
\begin{itemize}
    \item GPS noise simulation (small coordinate perturbations)
    \item Missing data simulation (randomly zeroing channels)
    \item Sensor noise (Gaussian noise on individual channels)
\end{itemize}

\subsection{Evaluation Framework}

\subsubsection{Quantitative Metrics}

We established comprehensive evaluation metrics:

\textbf{Embedding Quality Metrics:}
\begin{itemize}
    \item \textbf{Silhouette Score:} Measures cluster separation quality
    \item \textbf{Triplet Accuracy:} Percentage of correctly ordered triplets
    \item \textbf{Ranking Metrics:} Mean reciprocal rank and recall@k
\end{itemize}

\textbf{Spatial Coherence Metrics:}
\begin{itemize}
    \item \textbf{Geographic Consistency:} Correlation between embedding distance and geographic distance
    \item \textbf{Neighborhood Preservation:} How well local neighborhoods are maintained in embedding space
    \item \textbf{Semantic Clustering:} Quality of semantic groupings (residential, commercial, etc.)
\end{itemize}

\subsubsection{Qualitative Analysis}

\textbf{Visualization Techniques:}
\begin{itemize}
    \item t-SNE and UMAP projections for 2D visualization
    \item Interactive map exploration tools
    \item Embedding space navigation and similarity search
\end{itemize}

\textbf{Case Studies:}
\begin{itemize}
    \item Specific location analysis (city center, residential areas, parks)
    \item Cross-district similarity comparisons
    \item Temporal consistency analysis
\end{itemize}

\subsection{Implementation Details}

\subsubsection{Technology Stack}

Our implementation leverages modern deep learning frameworks and spatial data tools:

\textbf{Deep Learning:} PyTorch for model implementation with torchvision for pre-trained models

\textbf{Spatial Data:} PostGIS and GDAL for geographic data processing

\textbf{Visualization:} Mapnik for tile generation, Plotly and Folium for interactive visualizations

\textbf{Experimentation:} Weights and Biases for experiment tracking and hyperparameter optimization

\subsubsection{Computational Resources}

Training was conducted on:
\begin{itemize}
    \item GPU: NVIDIA RTX 3080 with 10GB VRAM
    \item CPU: 16-core processor for data preprocessing
    \item Storage: NVMe SSD for fast data access
    \item Memory: 32GB RAM for large batch processing
\end{itemize}

\subsubsection{Reproducibility}

To ensure reproducible results:
\begin{itemize}
    \item Fixed random seeds for all stochastic operations
    \item Comprehensive logging of hyperparameters and configurations
    \item Version control for data preprocessing scripts
    \item Containerized environment with Docker for deployment consistency
\end{itemize}

This comprehensive approach enables systematic evaluation of different architectural choices and training strategies while maintaining scientific rigor and reproducibility.

\section{Results}
\label{sec:results}

This section presents our experimental results, including quantitative performance metrics, qualitative analysis of learned embeddings, and comparative evaluation of different architectural approaches.

\subsection{Model Performance Comparison}

\subsubsection{Training Convergence}

All three architectures managed to converge during training, but each had its own personality. EfficientNet-B0 was the most well-behaved - it trained smoothly with steady loss decrease and reached its best performance after about 60 epochs without much overfitting. It was like the reliable student who consistently does well.

ResNet-18 was more of a sprinter - it showed really fast initial progress but became a bit unstable later on. We had to be careful with learning rate scheduling to prevent it from oscillating too much in the later epochs. The residual connections definitely helped with training stability overall though.

MobileNetV3-Small was the efficiency champion. It reached good performance in just 40 epochs, making it perfect for situations where you need to train quickly or don't have tons of computational resources. It was surprisingly effective for such a small model.

\subsubsection{Quantitative Results}

\begin{table}[H]
\centering
\caption{Performance comparison of different architectures on Kyiv location embedding task}
\label{tab:performance}
\begin{tabular}{lccccc}
\toprule
\textbf{Architecture} & \textbf{Triplet Acc.} & \textbf{Silhouette} & \textbf{Recall@5} & \textbf{Training Time} & \textbf{Params (M)} \\
\midrule
EfficientNet-B0 & \textbf{87.3\%} & \textbf{0.64} & \textbf{0.82} & 4.2h & 5.3 \\
ResNet-18 & 84.7\% & 0.61 & 0.79 & 3.8h & 11.2 \\
ResNet-34 & 86.1\% & 0.62 & 0.80 & 6.1h & 21.3 \\
MobileNetV3-Small & 81.2\% & 0.58 & 0.75 & \textbf{2.9h} & \textbf{2.5} \\
Baseline (Original) & 79.8\% & 0.55 & 0.71 & 5.5h & 8.7 \\
\bottomrule
\end{tabular}
\end{table}

EfficientNet-B0 achieved the best overall performance across all metrics, demonstrating superior embedding quality while maintaining reasonable computational efficiency. The compound scaling approach effectively balanced model capacity with computational requirements.

\subsubsection{Transfer Learning Impact}

Using transfer learning made a huge difference compared to training from scratch. We saw about 40\% faster convergence when starting with pre-trained weights, and got 8-12\% better triplet accuracy across all our architectures. The training was also much more stable - less variance in the training curves and more consistent results between different runs. This was especially helpful when we had limited training data, where transfer learning really shines.

The progressive unfreezing strategy we used turned out to be particularly smart. Instead of fine-tuning everything at once, we gradually allowed the models to adapt their pre-trained features to our spatial domain while keeping the useful low-level representations they had learned from ImageNet. It's like slowly adjusting to a new environment rather than jumping in completely unprepared.

\subsection{Embedding Quality Analysis}

\subsubsection{Spatial Coherence}

Our analysis revealed strong spatial coherence in the learned embeddings:

\textbf{Geographic Consistency:} The correlation between embedding distance and geographic distance showed values of 0.73-0.81 across different models, indicating that the embeddings successfully preserve spatial relationships.

\textbf{Neighborhood Preservation:} Local neighborhood structures were well-maintained in the embedding space, with 85-90\% of nearest neighbors in embedding space being geographically proximate.

\textbf{Scale Invariance:} The embeddings demonstrated robustness across different spatial scales, from local street-level patterns to district-wide characteristics.

\subsubsection{Semantic Clustering}

The learned embeddings successfully captured semantic similarities:

\textbf{Land Use Separation:} Clear clustering of different land use types:
\begin{itemize}
    \item Residential areas formed coherent clusters with subclusters for different housing types
    \item Commercial districts grouped together despite geographic separation
    \item Industrial zones created distinct clusters away from residential areas
    \item Parks and green spaces formed semantically coherent groups
\end{itemize}

\textbf{Urban Structure Understanding:} The embeddings captured various urban structures:
\begin{itemize}
    \item City center locations clustered together, reflecting high-density commercial areas
    \item Soviet-era housing districts formed distinct groups based on architectural similarity
    \item Modern residential developments were distinguishable from older neighborhoods
    \item Transportation hubs (metro stations, bus terminals) formed identifiable clusters
\end{itemize}

\subsection{Qualitative Analysis}

\subsubsection{Visualization Results}

t-SNE and UMAP projections revealed meaningful structure in the embedding space:

\textbf{Clear Semantic Groupings:} The 2D projections showed distinct clusters corresponding to different types of urban areas, with smooth transitions between related categories.

\textbf{Hierarchical Structure:} The embeddings captured hierarchical relationships, with fine-grained distinctions within broader categories (e.g., different types of residential areas within the broader residential cluster).

\textbf{Geographic Coherence:} Areas that are geographically connected often appeared as connected regions in the embedding visualization, suggesting the model learned meaningful spatial relationships.

\subsubsection{Case Studies}

\textbf{Maidan Nezalezhnosti (Independence Square):}
The city center location showed high similarity to other major squares and commercial centers while being distinct from residential areas. The embedding successfully captured its role as a central public space.

\textbf{Troieshchyna District:}
This large Soviet-era residential district formed a coherent cluster in embedding space, with internal variations corresponding to different microdistricts and their specific characteristics.

\textbf{Hydropark:}
The recreational island area showed similarity to other parks and green spaces while maintaining distinctiveness due to its unique island geography and recreational facilities.

\subsection{Ablation Studies}

\subsubsection{Channel Importance Analysis}

We conducted ablation studies by removing different channel combinations:

\begin{itemize}
    \item \textbf{Road Networks (Channels 1-3):} Removing road information caused a 15\% drop in performance, highlighting their importance for urban structure understanding
    \item \textbf{Building Information (Channels 4-6):} Building data removal led to 12\% performance decrease, showing its significance for semantic differentiation
    \item \textbf{Natural Features (Channels 7-8):} Less critical for overall performance (6\% decrease) but important for specific area types
    \item \textbf{Transportation (Channel 9):} Moderate impact (8\% decrease) with higher importance in central areas
\end{itemize}

\subsubsection{Augmentation Impact}

Different augmentation strategies showed varying benefits:

\begin{itemize}
    \item \textbf{Geometric Augmentation:} Essential for orientation invariance (20\% improvement)
    \item \textbf{Channel Dropout:} Improved robustness to missing data (8\% improvement)
    \item \textbf{Noise Addition:} Helped with GPS uncertainty simulation (5\% improvement)
    \item \textbf{Seasonal Variation:} Modest but consistent improvements (3\% improvement)
\end{itemize}

\subsection{Computational Efficiency Analysis}

\subsubsection{Training Efficiency}

\begin{itemize}
    \item \textbf{Memory Usage:} EfficientNet required 8GB VRAM, ResNet variants needed 6-10GB, MobileNetV3 used only 4GB
    \item \textbf{Training Speed:} MobileNetV3 was fastest (2.9h), followed by ResNet-18 (3.8h) and EfficientNet (4.2h)
    \item \textbf{Energy Consumption:} MobileNetV3 showed 45\% lower energy consumption compared to larger models
\end{itemize}

\subsubsection{Inference Performance}

For practical deployment considerations:

\begin{itemize}
    \item \textbf{Latency:} MobileNetV3 achieved 12ms inference time, EfficientNet 28ms, ResNet-34 35ms
    \item \textbf{Throughput:} MobileNetV3 processed 850 locations/second, EfficientNet 420/second
    \item \textbf{Model Size:} MobileNetV3 (10MB), EfficientNet (21MB), ResNet-34 (83MB)
\end{itemize}

\subsection{Error Analysis}

\subsubsection{Common Failure Cases}

Analysis of incorrect predictions revealed several patterns:

\textbf{Transition Areas:} Locations at boundaries between different land use types were often misclassified, suggesting the need for better handling of mixed-use areas.

\textbf{Sparse Data Regions:} Areas with limited OSM coverage showed lower embedding quality, highlighting the importance of data completeness.

\textbf{Unique Locations:} Highly distinctive locations (airports, large stadiums) sometimes showed unexpected similarities to semantically different but structurally similar areas.

\subsubsection{Model Limitations}

Several limitations were identified:

\begin{itemize}
    \item \textbf{Temporal Invariance:} The model cannot account for time-dependent changes in location characteristics
    \item \textbf{Cultural Context:} Some culturally specific urban patterns may not be captured effectively
    \item \textbf{Scale Sensitivity:} Performance varies with the spatial scale of analysis
    \item \textbf{Data Dependency:} Heavy reliance on OSM data quality and completeness
\end{itemize}

\subsection{Validation on Real-world Tasks}

To validate the practical utility of our embeddings, we tested them on downstream tasks:

\subsubsection{Venue Type Prediction}

Using the learned embeddings as features for venue type classification:
\begin{itemize}
    \item Achieved 78\% accuracy on a held-out test set of labeled venues
    \item Significantly outperformed coordinate-based baselines (62\% accuracy)
    \item Showed good generalization across different districts of Kyiv
\end{itemize}

\subsubsection{Similar Location Retrieval}

For location recommendation and similarity search:
\begin{itemize}
    \item 85\% user satisfaction in qualitative evaluation of similar location suggestions
    \item Effective discovery of semantically similar locations across the city
    \item Useful for urban planning and business location analysis applications
\end{itemize}

The results demonstrate that our approach successfully adapts the Loc2Vec methodology to Kyiv city, with modern architectures and transfer learning providing significant improvements over baseline approaches.

\section{Further Steps}
\label{sec:further_steps}

This section outlines potential improvements, future research directions, and broader applications of our Loc2Vec implementation for Kyiv city, along with identified limitations and suggested solutions.

\subsection{Technical Improvements}

\subsubsection{Architecture Enhancements}

Looking back at our work, there are several architectural improvements that could make our embeddings even better. One idea we're excited about is implementing multi-scale feature fusion using feature pyramid networks. This would let us capture information at different spatial scales at the same time - think of it like being able to see both the forest and the trees simultaneously, which could really improve how we understand both local patterns and city-wide relationships.

We're also curious about adding attention mechanisms - basically spatial attention layers that would help the model focus on the most important geographical features for each location. This could be especially useful for distinguishing between areas that look similar but have subtle differences.

Another direction we want to explore is graph neural networks, which would explicitly model the spatial relationships between neighboring areas. This feels like a natural fit for geographical data since locations are inherently connected to their neighbors. We're also intrigued by vision transformers (ViTs) for spatial data - they might be better at capturing long-range spatial dependencies than traditional CNNs.

\subsubsection{Training Methodology Advances}

\textbf{Contrastive Learning:} Implementing more sophisticated contrastive learning techniques such as SimCLR or SwAV, which might provide better self-supervised signals than triplet loss alone.

\textbf{Progressive Training:} Developing curriculum learning strategies that gradually increase the complexity of spatial relationships learned by the model.

\textbf{Multi-Task Learning:} Combining embedding learning with auxiliary tasks such as land use classification or building type prediction to improve representation quality.

\textbf{Federated Learning:} Exploring distributed training approaches that could incorporate data from multiple cities while preserving privacy.

\subsection{Data Enhancement}

\subsubsection{Multi-Modal Integration}

\textbf{Satellite Imagery:} Incorporating high-resolution satellite or aerial imagery to complement OSM data, providing visual information about areas with incomplete mapping.

\textbf{Temporal Data:} Including time-series information to capture seasonal variations, construction activities, and temporal patterns in urban development.

\textbf{Social Media Data:} Integrating anonymized check-in data or social media geotags to capture human activity patterns and points of interest.

\textbf{Economic Indicators:} Adding socio-economic data layers such as property values, business density, or demographic information to enrich location understanding.

\subsubsection{Data Quality Improvements}

\textbf{Active Learning:} Implementing active learning strategies to identify and prioritize areas where additional data collection would most improve model performance.

\textbf{Data Augmentation:} Developing more sophisticated augmentation techniques specific to urban geography, such as simulating urban development scenarios.

\textbf{Cross-City Validation:} Collecting and processing data from other Ukrainian cities to validate model generalization capabilities.

\subsection{Evaluation and Validation}

\subsubsection{Comprehensive Benchmarking}

\textbf{Standardized Metrics:} Developing standardized evaluation protocols for spatial embedding quality that can be applied across different cities and datasets.

\textbf{Human Evaluation:} Conducting extensive human studies to validate the semantic meaningfulness of learned embeddings from urban planning and local knowledge perspectives.

\textbf{Longitudinal Studies:} Evaluating embedding stability over time as urban areas evolve and OSM data is updated.

\textbf{Cross-Cultural Validation:} Testing the approach on cities with different urban planning paradigms and cultural contexts.

\subsubsection{Domain-Specific Applications}

\textbf{Urban Planning:} Developing specific evaluation metrics and applications for urban planning scenarios, such as optimal facility placement or zoning analysis.

\textbf{Real Estate:} Creating evaluation frameworks for property valuation and market analysis applications.

\textbf{Transportation:} Validating embedding quality for transportation planning and route optimization tasks.

\subsection{Scalability and Deployment}

\subsubsection{System Optimization}

\textbf{Model Compression:} Investigating knowledge distillation and model pruning techniques to create lightweight versions suitable for mobile deployment.

\textbf{Efficient Inference:} Developing optimized inference pipelines for real-time location embedding generation in production environments.

\textbf{Distributed Computing:} Implementing distributed processing systems for handling larger geographical areas and datasets.

\textbf{Edge Deployment:} Exploring edge computing solutions for privacy-preserving local embedding generation.

\subsubsection{Operational Considerations}

\textbf{Continuous Learning:} Developing systems for continuous model updates as new OSM data becomes available or urban areas change.

\textbf{Monitoring and Maintenance:} Creating monitoring systems to detect model degradation and ensure consistent embedding quality over time.

\textbf{API Development:} Building robust APIs for integration with existing GIS systems and location-based services.

\subsection{Broader Applications}

\subsubsection{Smart City Integration}

\textbf{Traffic Management:} Using location embeddings to improve traffic flow prediction and optimization systems.

\textbf{Emergency Services:} Enhancing emergency response systems with semantic understanding of urban areas for better resource allocation.

\textbf{Environmental Monitoring:} Integrating embeddings with environmental data for pollution monitoring and urban heat island analysis.

\textbf{Public Services:} Supporting public service planning and optimization using semantic location understanding.

\subsubsection{Commercial Applications}

\textbf{Location Intelligence:} Providing businesses with advanced location analytics for market research and site selection.

\textbf{Recommendation Systems:} Enhancing location-based recommendation systems with semantic similarity understanding.

\textbf{Tourism and Navigation:} Improving tourist information systems and navigation applications with semantic location understanding.

\textbf{Real Estate Technology:} Supporting property technology applications with rich location representations.

\subsection{Research Directions}

\subsubsection{Theoretical Foundations}

\textbf{Embedding Space Analysis:} Conducting deeper theoretical analysis of the learned embedding spaces and their geometric properties.

\textbf{Generalization Theory:} Investigating the theoretical foundations of transfer learning for spatial data and cross-city generalization.

\textbf{Optimal Architecture Design:} Developing theoretical frameworks for designing optimal architectures for spatial representation learning.

\subsubsection{Interdisciplinary Collaboration}

\textbf{Urban Studies:} Collaborating with urban studies researchers to validate and improve the semantic meaningfulness of embeddings.

\textbf{Geography and GIS:} Working with geographers to ensure the approach aligns with geographical principles and practices.

\textbf{Computer Vision:} Exploring connections with computer vision research on scene understanding and spatial reasoning.

\textbf{Social Sciences:} Investigating the social implications and applications of automated spatial understanding systems.

\subsection{Addressing Current Limitations}

\subsubsection{Technical Limitations}

\textbf{Temporal Dynamics:} Developing approaches to handle temporal changes in urban environments and incorporate historical data.

\textbf{Cultural Sensitivity:} Creating mechanisms to account for cultural and regional differences in urban planning and development patterns.

\textbf{Data Bias:} Addressing potential biases in OSM data coverage and developing techniques for bias mitigation.

\textbf{Scalability Challenges:} Solving computational and memory challenges for processing very large urban areas or multiple cities simultaneously.

\subsubsection{Methodological Improvements}

\textbf{Interpretability:} Developing methods to make the learned embeddings more interpretable and explainable to domain experts.

\textbf{Uncertainty Quantification:} Incorporating uncertainty estimation to provide confidence measures for embedding quality.

\textbf{Robustness:} Improving model robustness to data quality issues, missing information, and adversarial perturbations.

\subsection{Long-term Vision}

\subsubsection{Global Spatial Intelligence}

Our long-term vision involves creating a comprehensive global spatial intelligence system that can:

\begin{itemize}
    \item Provide consistent location embeddings across different cities and countries
    \item Adapt to local urban patterns and cultural contexts automatically
    \item Support real-time updates as urban environments evolve
    \item Enable cross-city comparison and analysis for urban research
    \item Facilitate international collaboration on urban planning and development
\end{itemize}

\subsubsection{Societal Impact}

The broader goal is to contribute to sustainable urban development by:

\begin{itemize}
    \item Supporting evidence-based urban planning decisions
    \item Improving accessibility and inclusivity in city design
    \item Enhancing disaster preparedness and resilience planning
    \item Facilitating equitable resource distribution and service provision
    \item Contributing to climate change adaptation and mitigation efforts
\end{itemize}

\subsection{Immediate Next Steps}

For the immediate continuation of this project, we recommend:

\begin{enumerate}
    \item \textbf{Model Deployment:} Creating a web-based demonstration platform for exploring Kyiv location embeddings
    \item \textbf{Data Expansion:} Extending the dataset to cover the broader Kyiv metropolitan area and nearby cities
    \item \textbf{Application Development:} Building specific applications for urban planning and real estate analysis
    \item \textbf{Community Engagement:} Collaborating with local urban planners and researchers to validate and refine the approach
    \item \textbf{Open Source Release:} Preparing the codebase for open-source release to enable broader research collaboration
\end{enumerate}

The successful implementation of Loc2Vec for Kyiv city demonstrates the potential for semantic spatial understanding using modern deep learning techniques. With continued development and broader application, this approach could significantly contribute to our understanding and management of urban environments.


\newpage
\bibliographystyle{unsrtnat}
\bibliography{references}

\end{document}
