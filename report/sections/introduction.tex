\section{Introduction}

In today's digital world, understanding where things happen has become incredibly important. From GPS navigation to urban planning and even targeted advertising, location-based services are everywhere. But here's the thing - most current approaches to representing locations are pretty basic. They either just use coordinates (latitude and longitude) or rely on manually crafted features, which don't really capture what makes different places similar or different.

Think about it this way: two places might be right next to each other on a map but be completely different in character - like a quiet residential street next to a busy industrial area. On the flip side, two shopping districts on opposite sides of a city might feel very similar even though they're geographically far apart. Traditional location representation methods struggle with this kind of semantic understanding.

The Loc2Vec methodology, originally developed by Sentiance \cite{sentiance2018loc2vec}, addresses this challenge by learning dense vector representations of geographical locations that capture both spatial and semantic relationships. The approach transforms locations into embeddings using convolutional neural networks trained with triplet loss, where the network learns to place semantically similar locations closer together in the embedding space.

\subsection{Problem Statement}

For our project, we decided to implement and adapt the Loc2Vec methodology specifically for Kyiv, Ukraine's capital city. This wasn't just a straightforward implementation though - we faced several interesting challenges. First, we wanted our model to truly understand what makes different areas of Kyiv unique, whether it's the historic city center, Soviet-era housing blocks, or modern business districts. We also needed to figure out how to encode the complex spatial relationships between different parts of the city.

What made our project particularly interesting was our decision to experiment with modern deep learning architectures that weren't available when the original Loc2Vec paper was published. We tested whether transfer learning with architectures like EfficientNet, ResNet, and MobileNetV3 could improve upon the original results. Finally, we had to adapt the entire approach to work well with Kyiv's specific urban characteristics and the available OpenStreetMap data for the city.

\subsection{Project Objectives}

Our main goals for this project were pretty ambitious. We wanted to build a complete working implementation of Loc2Vec from scratch, using OpenStreetMap data specifically for Kyiv. But we didn't stop there - we were curious about whether modern neural network architectures could do better than the original approach, so we planned to test EfficientNet, ResNet, and MobileNetV3 as potential encoders.

We were also really interested in transfer learning. The idea was to see if starting with pre-trained models (instead of training from scratch) could make our training faster and produce better embeddings. To make sure we could properly evaluate our work, we needed to set up comprehensive metrics to measure how good our location embeddings actually were. And of course, we wanted to show that these embeddings could be useful for real applications, not just academic exercises.

\subsection{Contributions}

Our work makes several contributions to the field of spatial representation learning:

\begin{itemize}
    \item A complete open-source implementation of Loc2Vec adapted for Kyiv city using modern deep learning frameworks
    \item Comprehensive comparison of different CNN architectures for location embedding learning
    \item Investigation of transfer learning techniques in the context of geographical data
    \item Analysis of the learned embeddings' ability to capture urban semantics and spatial relationships
    \item Practical insights into the application of location embeddings for real-world problems
\end{itemize}

\subsection{Report Structure}

This report is organized as follows: Section \ref{sec:related_work} reviews relevant literature in spatial representation learning and embedding techniques. Section \ref{sec:data} describes our data collection and preprocessing pipeline for Kyiv city. Section \ref{sec:approach} details our implementation methodology and architectural choices. Section \ref{sec:results} presents experimental results and analysis. Finally, Section \ref{sec:further_steps} discusses future directions and potential improvements.
