This report presents our implementation and analysis of the Loc2Vec approach for learning location embeddings using triplet-loss networks, specifically applied to Kyiv city data. Loc2Vec is a deep learning solution that encodes geo-spatial relations and semantic similarities of location surroundings to improve venue mapping accuracy. We reimplemented the original methodology proposed by Sentiance, adapting it for Ukrainian geographical data and urban characteristics specific to Kyiv.

Our implementation involved creating a comprehensive pipeline for processing OpenStreetMap data, generating multi-channel rasterized representations of geographical areas, and training convolutional neural networks using triplet loss to learn meaningful location embeddings. Beyond the original implementation, we conducted extensive experiments with transfer learning approaches, comparing the performance of EfficientNet, ResNet, and MobileNetv3 architectures as backbone encoders.

The project demonstrates that location embeddings can effectively capture semantic similarities between geographical areas in Kyiv, enabling improved venue mapping and location understanding. Our transfer learning analysis reveals varying performance characteristics across different pre-trained architectures, with EfficientNet showing promising results for balancing accuracy and computational efficiency. The work contributes to the field of geographical machine learning by providing insights into adapting location embedding techniques for Eastern European urban environments and evaluating modern CNN architectures for geographical data processing.

Key contributions include: (1) successful adaptation of Loc2Vec for Kyiv city data, (2) comprehensive evaluation of transfer learning approaches for location embedding, (3) analysis of computational trade-offs between different CNN architectures, and (4) insights into geographical representation learning for Ukrainian urban landscapes.
