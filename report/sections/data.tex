\section{Data}
\label{sec:data}

This section describes our data collection and preprocessing pipeline for creating location embeddings specific to Kyiv city. We detail the data sources, preprocessing steps, and the challenges encountered in adapting the methodology to the Ukrainian capital.

\subsection{Data Sources}

\subsubsection{OpenStreetMap Data}

We decided to use OpenStreetMap (OSM) as our main data source, which turned out to be a great choice. OSM is basically Wikipedia for maps - it's a collaborative project where people around the world contribute geographical information. For Kyiv specifically, we were able to extract a ton of useful data including the complete road network (everything from tiny residential streets to major highways), building outlines for different types of structures, and land use information that tells us where the parks, industrial areas, and residential zones are located.

We also got detailed information about points of interest like restaurants, shops, schools, hospitals, and cultural sites, plus all the transportation infrastructure including metro stations, bus stops, and railway stations. One of the coolest things about working with Kyiv data was getting to map natural features like the Dnieper River, which runs right through the city, along with various lakes and green spaces.

What really helped us was that Kyiv has a pretty active OpenStreetMap community, so the data quality was surprisingly good. This gave us detailed coverage of the city's infrastructure and amenities that we might not have gotten in other cities.

\subsubsection{Administrative Boundaries}

We incorporated administrative boundary data to understand the city's structure:

\begin{itemize}
    \item \textbf{District Boundaries:} Kyiv's 10 administrative districts (raions)
    \item \textbf{Neighborhood Boundaries:} Historical and informal neighborhood divisions
    \item \textbf{City Limits:} Official boundaries of Kyiv and surrounding areas
\end{itemize}

\subsection{Geographic Coverage}

Our dataset covers the greater Kyiv metropolitan area, including:

\begin{itemize}
    \item \textbf{Central Kyiv:} Historical city center and main commercial districts
    \item \textbf{Residential Areas:} Soviet-era housing districts and modern developments
    \item \textbf{Industrial Zones:} Manufacturing and industrial areas
    \item \textbf{Suburban Areas:} Surrounding suburbs and satellite towns
    \item \textbf{Natural Areas:} Parks, forests, and the Dnieper River corridor
\end{itemize}

The total area covered spans approximately 1,500 square kilometers, encompassing the diverse urban landscape of Kyiv from dense city centers to suburban and semi-rural areas.

\subsection{Data Preprocessing Pipeline}

\subsubsection{Geographic Information System Setup}

We established a comprehensive GIS processing pipeline:

\begin{enumerate}
    \item \textbf{Database Setup:} Installed PostgreSQL with PostGIS extensions for spatial data operations
    \item \textbf{OSM Import:} Used osm2pgsql to import Kyiv OSM data into the spatial database
    \item \textbf{Data Validation:} Performed quality checks and cleaned inconsistent or incomplete data
    \item \textbf{Indexing:} Created spatial indices for efficient geographic queries
\end{enumerate}

\subsubsection{Map Tile Generation}

Following the original Loc2Vec approach, we developed a map rasterization system:

\textbf{Rasterization Framework:} We used Mapnik with custom stylesheets to convert vector geographic data into raster tiles. This process generates multi-channel tensors where each channel represents a specific type of geographical feature.

\textbf{Channel Configuration:} Our 12-channel tensor representation includes:
\begin{itemize}
    \item Channel 1: Major roads and highways
    \item Channel 2: Secondary and residential roads
    \item Channel 3: Pedestrian paths and cycling routes
    \item Channel 4: Residential buildings
    \item Channel 5: Commercial and office buildings
    \item Channel 6: Industrial buildings
    \item Channel 7: Water bodies (rivers, lakes)
    \item Channel 8: Green spaces (parks, forests)
    \item Channel 9: Public transportation infrastructure
    \item Channel 10: Points of interest and amenities
    \item Channel 11: Administrative boundaries
    \item Channel 12: Land use classifications
\end{itemize}

\textbf{Tile Specifications:} Each tile represents a 500m × 500m area at 256×256 pixel resolution, providing sufficient detail to capture local geographical structures while maintaining computational efficiency.

\subsection{Data Augmentation}

To increase dataset diversity and improve model generalization, we implemented several data augmentation techniques:

\subsubsection{Geometric Transformations}
\begin{itemize}
    \item \textbf{Rotation:} Random rotations from 0° to 360° to ensure orientation invariance
    \item \textbf{Translation:} Random horizontal and vertical shifts up to 50 pixels
    \item \textbf{Scaling:} Minor zoom variations (±10\%) to account for different detail levels
\end{itemize}

\subsubsection{Channel-wise Augmentation}
\begin{itemize}
    \item \textbf{Noise Addition:} Gaussian noise to simulate GPS uncertainty and mapping errors
    \item \textbf{Channel Dropout:} Randomly disabling certain channels to improve robustness
    \item \textbf{Intensity Variations:} Adjusting channel intensities to simulate different data qualities
\end{itemize}

\subsection{Dataset Statistics}

Our final dataset contains:

\begin{itemize}
    \item \textbf{Total Locations:} 50,000 unique coordinate pairs covering Kyiv
    \item \textbf{Tile Coverage:} Complete coverage of urban areas with 80\% coverage of suburban areas
    \item \textbf{Training Set:} 35,000 locations (70\%)
    \item \textbf{Validation Set:} 7,500 locations (15\%)
    \item \textbf{Test Set:} 7,500 locations (15\%)
    \item \textbf{Augmented Samples:} 5x augmentation resulting in 250,000 training samples
\end{itemize}

\subsection{Triplet Generation Strategy}

For triplet loss training, we developed a sophisticated triplet mining strategy:

\subsubsection{Positive Pair Generation}
Positive pairs are locations that should have similar embeddings:
\begin{itemize}
    \item \textbf{Spatial Proximity:} Locations within 100-200m radius
    \item \textbf{Semantic Similarity:} Locations with similar land use patterns
    \item \textbf{Infrastructure Similarity:} Areas with comparable road networks and building densities
\end{itemize}

\subsubsection{Negative Pair Generation}
Negative pairs are semantically different locations:
\begin{itemize}
    \item \textbf{Different Districts:} Locations from different administrative areas
    \item \textbf{Contrasting Land Use:} Industrial vs. residential, urban vs. natural areas
    \item \textbf{Infrastructure Differences:} Dense urban areas vs. sparse suburban areas
\end{itemize}

\subsubsection{Hard Negative Mining}
We implemented online hard negative mining to focus training on challenging examples:
\begin{itemize}
    \item \textbf{Semi-hard Negatives:} Negatives that are closer to the anchor than the positive
    \item \textbf{Dynamic Selection:} Updating negative selection based on current model performance
    \item \textbf{Balanced Sampling:} Ensuring representation across different geographical areas
\end{itemize}

\subsection{Data Quality and Challenges}

\subsubsection{Quality Assurance}
We implemented several quality control measures:
\begin{itemize}
    \item \textbf{Visual Inspection:} Manual review of generated tiles for accuracy
    \item \textbf{Consistency Checks:} Verification of channel alignment and data integrity
    \item \textbf{Boundary Validation:} Ensuring proper handling of city boundaries and water bodies
\end{itemize}

\subsubsection{Challenges Encountered}
Several challenges were addressed during data preparation:
\begin{itemize}
    \item \textbf{Data Completeness:} Some areas had incomplete OSM coverage requiring additional validation
    \item \textbf{Urban Complexity:} Kyiv's diverse architecture and Soviet-era planning required careful feature representation
    \item \textbf{Seasonal Variations:} OSM data represents a snapshot in time, not accounting for seasonal changes
    \item \textbf{Scale Consistency:} Ensuring consistent representation across different urban densities
\end{itemize}

The resulting dataset provides a comprehensive foundation for learning meaningful location embeddings that capture the unique characteristics of Kyiv's urban landscape.
